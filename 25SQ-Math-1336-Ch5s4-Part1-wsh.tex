% Instructions to change to html version:
% Comment out:
%  minipage, multicols,columnbreak, mathbf, hrule
% Replace all: \begin{minipage}% \end{minipage} %\begin{multicols}  %\end{multicols} 
% \columnbreak
 %	%% \begin{framed} %\end{framed} %%\hrule
% Replace \mathbf with	\boldsymbol
% Replace $$ with \[ or \]and $ with \( or \)
% Enclose graphics in figure environments and add captions
% 			search \includegraphics
% Re-tag \df environments as sections, subsections, etc.
% Command Line Code to Create html version:
%First: pdflatex -shell-escape filename.tex                                   
%Second, for each figure: inkscape "filename-figure1.pdf" -o "filename-figure1.png"
% Third: htlatex filename.tex "ht5mjlatex.cfg, charset=utf-8" " -cunihtf -utf8"

\documentclass[10pt]{article}

%\usepackage{tikz, pgf,pgfplots,wasysym,array}
%\usepackage{wasysym,array}

\usepackage{amsmath,amssymb}
\usepackage[hidelinks]{hyperref}

\ifdefined\HCode
  \def\pgfsysdriver{pgfsys-tex4ht-updated.def}
\fi 
%\ifdefined\HCode
%  \def\pgfsysdriver{pgfsys-dvisvgm4ht.def}
%\fi 
\usepackage{tikz}
\usetikzlibrary{calc,decorations.markings,arrows}
\usepackage{pgfplots}

\pgfplotsset{compat=1.12}
\usepackage{myexternalize}
\usetikzlibrary{calc,decorations.markings,arrows}
\usepackage{framed}
\usepackage[none]{hyphenat}

\input{../../../common/1336_header_test.tex}
\begin{document}

\newcommand{\an}{\lbrace a_n \rbrace}
\newcommand{\Sum}{\sum_{n=1}^\infty }

\everymath{\displaystyle}
%\vspace*{-.25in}
\renewcommand{\myTitle}{	MATH 1336: Calculus III}

\renewcommand{\mySubTitle}{Section 5.4, Part 1: Comparison Tests}% \vspace*{-.25in}}
%~\hfill Name: \underline{~~~~~~~~~~~~~~~~~~~~~~~~~~~~~~~~~~~~~~~~~~~~~~~}


\lectTitle{\vspace*{-.6in}\myTitle}{\vspace*{.1in}\mySubTitle \vspace*{-.3in}}


%\hspace*{-.8in}%\begin{minipage}{1.25\textwidth}

\setlength{\columnseprule}{.4pt}
\setlength{\columnsep}{3em}

%\begin{framed}
\section*{Integral \& Comparison Tests - Tests for Series with POSITIVE Terms: }
\textbf{These tests can only be applied to series with \textit{POSITIVE} terms: \(\boldsymbol{a_n >0}\)}\\
\(\Rightarrow\)  verifying and stating that \(\boldsymbol{a_n >0}\) is an important part of the argument when using these tests!\\
%\boldsymbol
\vspace*{.1in}

%\begin{multicols}{2}


\section*{Comparison Test:}
%Given that \(\sum a_n\) and \(\sum b_n\) are series with \textbf{positive} terms, then we have the following:
\begin{enumerate}[(i)]
\item If \(\sum b_n\) is convergent and \(0\leq a_n \leq b_n\) for all \(n\geq N\),\\ then \(\sum a_n\) is also convergent.
\item If \(\sum b_n\) is divergent and \(0\leq b_n \leq a_n\) for all \(n\geq N\),\\ then \(\sum a_n\) is also divergent.
\end{enumerate}

%\columnbreak

\section*{Limit Comparison Test:}
%Given that \(\sum a_n\) and \(\sum b_n\) are series with \textbf{positive} terms, if:
Let \(a_n, b_n \geq 0\) for all \(n\geq 1\).
\begin{enumerate}[(i)]
\item If \(\lim_{n\rightarrow \infty} \frac{a_n}{b_n} = c,\) \textbf{and}  \(0<c<\infty\), 
then \(\sum a_n\) and \(\sum b_n\)\\
 have the \textbf{same convergence behavior}.\\ %either they both converge or they both diverge.\
 \item If \(\lim_{n\rightarrow \infty} \frac{a_n}{b_n} = 0,\) \textbf{and}
 \(\sum b_n\) converges,\\ then \(\sum a_n\) also converges.\\
  \item If \(\lim_{n\rightarrow \infty} \frac{a_n}{b_n} = \infty,\) \textbf{and}
 \(\sum b_n\) diverges,\\ then \(\sum a_n\) also diverges.\\
\end{enumerate}
%\[
%\lim_{n\rightarrow \infty} \frac{a_n}{b_n} = c,
%\]
%where \(c>0\) is \textbf{finite}, then either the series \textbf{both converge} or they \textbf{both diverge}.


%\end{multicols}

%\boldsymbol
\vspace*{.2in}
\setlength{\columnseprule}{0pt}
\setlength{\columnsep}{1em}

\textbf{Note 1 - What to compare with?}\\
Both tests rely on comparison with series for which we already know the convergence behavior:
%\begin{multicols}{4}
\begin{itemize}
\item Geometric Series
\item Harmonic Series
\item p-Series
\item Series we can test another way...
\end{itemize}
%\end{multicols}

\textbf{Note 2 - \(\boldsymbol{n\geq N}\)}\\
We don't \textit{always} have to start with \(n=1\). Convergence is really about the tails, or end behavior, of the terms.\\ If we can find a comparison that only holds for \(n\geq 500\), for example, the tests still work!

%\end{framed}

%\end{minipage}

\section*{Motivating Example (pre-class video):}
\[
\text{\textbf{Intuition:}} \qquad  \Sum \frac{1}{3^n +1} \qquad \text{and} \qquad \Sum \frac{1}{3^n} \qquad \text{should have the \textbf{same convergence behavior}}
\]
~\\

We \textit{proved} it by observing that:\\
\( 0 <  \frac{1}{3^n +1} <  \frac{1}{3^n}\) for \(n\geq 1\). 
 Then
we used the \(n^{th}\) partial sum formula for Geometric Series to show that %, \(s_n = \frac{a(1-r^n)}{1-r}\), with \(r=\tfrac{1}{3}\) to show %to get an upper bound on the \(n^{th}\) partial sum of the series \(\Sum \frac{1}{3^n +1}\):
 \\
%\( 0 \leq s_n \leq \frac{\tfrac{1}{3}(1-(\tfrac{1}{3})^n)}{1-\tfrac{1}{3}}\), then in the limit as \(n\rightarrow \infty\)\\
\( 0 < \lim_{n\rightarrow \infty} s_n < \frac{1}{2}\), so the sequence of partial sums \(\lbrace s_n\rbrace\) is \textit{bounded}.\\

We also know that \(\lbrace s_n\rbrace\) is increasing, since \( \frac{1}{3^n +1}>0\), so the series must converge by the \textbf{MCT}.\\

The Comparison Tests provide a faster way to use the same intuition that can also be applied to more situations!

\pagebreak

\section*{Examples:}
We will work through the following examples together.

\textbf{Do the series listed below converge, diverge, or are we unable to determine the convergence behavior given the tools that we have?}

\begin{enumerate}[{Example }1:]

%\addtocounter{enumi}{1}

\item \(\qquad \sum_{n=1}^\infty \frac{1}{3^n +1}\) \vfill

\item \(\qquad \sum_{n=2}^\infty \frac{1}{\sqrt{n}-1}\) \vfill

\item \(\qquad \sum_{n=1}^\infty \frac{1}{n!}\) \vfill

\item \(\qquad \sum_{n=1}^\infty \frac{1}{3^n-1}\) \vfill


\end{enumerate}

\pagebreak

\section*{Problems for Group Work:}
\textbf{Be sure to fully justify your reasoning as a part of your solutions.}\\
 The answers are upside-down on the bottom of this page.

%\begin{framed}
For Problems \ref{prob1}-\ref{prob4}, use either the Comparison Test or the Limit Comparison Test to determine whether the series is convergent or divergent. If neither of the tests can be used, explain why.
%\end{framed}

%\begin{multicols}{2}
\begin{enumerate}[{Problem }1:]

\item \(\qquad \Sum \frac{1}{\sqrt{n}\ 3^n}\) \label{prob1}
%Converge
\vfill

\item \(\qquad\Sum \frac{1}{\sqrt[3]{n}+1}\) \label{prob2}
%Diverge
\vfill

\item \(\qquad\Sum \frac{\sin(n)}{n^2}\) \label{prob3}
%Cannot Determine
\vfill

\item \(\qquad\Sum \frac{1}{5^n +300}\) \label{prob4}
%Converge
\vfill

\end{enumerate}
%\end{multicols}
%\vfill

\rotatebox{180}{
%\begin{minipage}{\textwidth}
\underline{Answers:}\\\textbf{Problem \ref{prob1}:} Converge, 
\textbf{Problem \ref{prob2}:} Diverge, 
\textbf{Problem \ref{prob3}:} Cannot Determine, 
\textbf{Problem \ref{prob4}:} Converge\\

%\end{minipage}
}

\end{document}
