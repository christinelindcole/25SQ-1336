% Instructions to change to html version:
% Comment out:
%  minipage, multicols,columnbreak, mathbf, hrule
% Replace all: \begin{minipage}% \end{minipage} %\begin{multicols}  %\end{multicols} 
% \columnbreak
 %	%% \begin{framed} %\end{framed} %%\hrule
% Replace \mathbf with	\boldsymbol
% Replace $$ with \[ or \]and $ with \( or \)
% Enclose graphics in figure environments and add captions
% 			search \includegraphics
% Re-tag \df environments as sections, subsections, etc.
% Command Line Code to Create html version:
%First: pdflatex -shell-escape filename.tex                                   
%Second, for each figure: inkscape "filename-figure1.pdf" -o "filename-figure1.png"
% Third: htlatex filename.tex "ht5mjlatex.cfg, charset=utf-8" " -cunihtf -utf8"

\documentclass[10pt]{article}

%\usepackage{tikz, pgf,pgfplots,wasysym,array}
%\usepackage{wasysym,array}

\usepackage{amsmath,amssymb}
\usepackage[hidelinks]{hyperref}

\ifdefined\HCode
  \def\pgfsysdriver{pgfsys-tex4ht-updated.def}
\fi 
%\ifdefined\HCode
%  \def\pgfsysdriver{pgfsys-dvisvgm4ht.def}
%\fi 
\usepackage{tikz}
\usetikzlibrary{calc,decorations.markings,arrows}
\usepackage{pgfplots}

\pgfplotsset{compat=1.12}
\usepackage{myexternalize}
\usetikzlibrary{calc,decorations.markings,arrows}
\usepackage{framed}
\usepackage[none]{hyphenat}

\input{../../../common/1336_header_test.tex}
\begin{document}



\newcommand{\an}{\lbrace a_n \rbrace}
\newcommand{\Sum}{\sum_{n=1}^\infty }
\newcommand{\Sumzero}{\sum_{n=0}^\infty }

\everymath{\displaystyle}

\renewcommand{\myTitle}{	MATH 1336: Calculus III}

\renewcommand{\mySubTitle}{Section 6.2: Fun with Power Series! }
%~\hfill Name: \underline{~~~~~~~~~~~~~~~~~~~~~~~~~~~~~~~~~~~~~~~~~~~~~~~}

\lectTitle{\vspace*{-.5in}\myTitle}{\vspace*{.1in}\mySubTitle \vspace*{-.2in}}

%\rfoot{{\texttt{Page \thepage~of 1}}}


%\hspace*{-.8in}%\begin{minipage}{1.25\textwidth}

\setlength{\columnseprule}{.4pt}
\setlength{\columnsep}{3em}

%\begin{framed}
%\textbf{New Tests \& Theorems:} 

%\begin{multicols}{2}
\section*{Section 6.2: Representing Functions with Power Series, AKA: Fun with Power Series!}
The theorem below tells us that we can treat Power Series just like more familiar functions, as long as we stay within the radius of convergence.
This means that if we know a given power series represents a particular function, we can manipulate it to find a power series representation for a new function!\\



\hrule
\vspace*{.2in}

%\begin{minipage}{.7\textwidth}
\subsection*{Power Series Differentiation \& Integration Theorem:}
If the power series, \(\Sumzero c_n (x-a)^n\) has a radius of convergence \(R>0\), then the function
\[
f(x) = c_0 + c_1(x-a) + c_2(x-a)^2 + \ldots = \Sumzero c_n (x-a)^n
\]
is differentiable (and therefore continuous!) on the interval \(I=(a-R, a+R)\) and
 \begin{enumerate}[(i)] 
 \item \(
f'(x) = c_1 + 2c_2(x-a) + 3c_3(x-a)^2 + \ldots = \Sum n c_n (x-a)^{n-1}
\)
 \item \(
\int f(x)\ dx = C + c_0(x-a) + c_1\frac{(x-a)^2}{2} + c_2\frac{(x-a)^3}{3} + \ldots = C+\Sumzero c_n \frac{(x-a)^{n+1}}{n+1}
\)
 \end{enumerate}
The radius of convergence for the series in (i) and (ii) is also \(R\).\\
%\end{minipage}


\hrule
\vspace*{.2in}

Our first building block will be the Geometric Series, because we know how to find the sum of that series. A surprising number of new series can be found using that building block, but the Geometric Series can only get us so far. We will need a more systematic method for finding new series, which we will learn in the next section.

%\begin{minipage}{.7\textwidth}
%\callout{
\[
\text{Geometric Series:}\qquad  \frac{1}{1-x}  = \sum_{n=0}^\infty x^{n}= 1+x+x^2+x^3+x^4+\ldots , \quad R=1
\]

%}
%\end{minipage}
\hspace*{.25in}
%\begin{minipage}{.25\textwidth}
\subsection*{Other Moves:}
\begin{itemize}
\item Add or Subtract Series
\item Multiply Series by a Scalar
\item Plug something in for \(x\)
\begin{itemize}
\item may change the radius of convergence
\end{itemize}
\end{itemize}

%\end{minipage}

%\end{framed}

%\end{minipage}

%\vfill

\section*{Warm Up / Brainstorming:}
%In the following examples, we will find power series representations for the functions listed below. 
To build intuition for the following examples, spend a few minutes brainstorming what you know about each function. %\\
\textit{Potential Topics: Domain? Range? Graph? Other useful facts?}

%\begin{multicols}{2}

\( \ln(x+1) \)

\( \arctan(x) \)

%\end{multicols}

\pagebreak

%\callout{
\[
\text{Geometric Series:}\qquad  \frac{1}{1-x}  = \sum_{n=0}^\infty x^{n}= 1+x+x^2+x^3+x^4+\ldots , \quad R=1
\]

%}

\section*{Examples:}



\begin{enumerate}[{Example } 1:]
\addtocounter{enumi}{3}

\item Discover a power series representation for \(\ln|1+x|\) using the geometric series as a building block.
\vfill


\item Discover a power series representation for \(\arctan(x)\) using the geometric series as a building block.
\vfill

%\callout{
%\[
%\mathrm{Geometric\ Series:}\qquad  \frac{1}{1-x}  = \sum_{n=0}^\infty x^{n}= 1+x+x^2+x^3+x^4+\ldots , \quad R=1
%\]
%
%}
\end{enumerate}

%\pagebreak
%
%\section*{Section 8.5, Part 2: Power Series Practice:}
%
%\begin{enumerate}[{Problem }1:]
%\item Find the radius and interval of convergence for the following power series:
%
%%\begin{multicols}{2}
%\begin{enumerate}
%
%\item \(\qquad \Sumzero \frac{(-5)^n (x-10)^n}{n!}\)
%
%\vfill%\vspace*{.5in}
%
%\item \(\qquad \Sumzero e^n(x-2)^n\)
%
%\vfill%\vspace*{.5in}
%
%\item \(\qquad \Sum \frac{(-2)^n x^n}{n^2}\)
%
%\vfill%\vspace*{.5in}
%
%\item \(\qquad \Sumzero x^n\)
%
%\vfill%\vspace*{.5in}
%
%\end{enumerate}
%%\end{multicols}
%%\vfill
%
%\end{enumerate}

%\vspace*{-.5in}
\end{document}
